% Options for packages loaded elsewhere
\PassOptionsToPackage{unicode}{hyperref}
\PassOptionsToPackage{hyphens}{url}
\PassOptionsToPackage{dvipsnames,svgnames,x11names}{xcolor}
%
\documentclass[
  letterpaper,
  DIV=11,
  numbers=noendperiod]{scrartcl}

\usepackage{amsmath,amssymb}
\usepackage{iftex}
\ifPDFTeX
  \usepackage[T1]{fontenc}
  \usepackage[utf8]{inputenc}
  \usepackage{textcomp} % provide euro and other symbols
\else % if luatex or xetex
  \usepackage{unicode-math}
  \defaultfontfeatures{Scale=MatchLowercase}
  \defaultfontfeatures[\rmfamily]{Ligatures=TeX,Scale=1}
\fi
\usepackage{lmodern}
\ifPDFTeX\else  
    % xetex/luatex font selection
\fi
% Use upquote if available, for straight quotes in verbatim environments
\IfFileExists{upquote.sty}{\usepackage{upquote}}{}
\IfFileExists{microtype.sty}{% use microtype if available
  \usepackage[]{microtype}
  \UseMicrotypeSet[protrusion]{basicmath} % disable protrusion for tt fonts
}{}
\makeatletter
\@ifundefined{KOMAClassName}{% if non-KOMA class
  \IfFileExists{parskip.sty}{%
    \usepackage{parskip}
  }{% else
    \setlength{\parindent}{0pt}
    \setlength{\parskip}{6pt plus 2pt minus 1pt}}
}{% if KOMA class
  \KOMAoptions{parskip=half}}
\makeatother
\usepackage{xcolor}
\setlength{\emergencystretch}{3em} % prevent overfull lines
\setcounter{secnumdepth}{-\maxdimen} % remove section numbering
% Make \paragraph and \subparagraph free-standing
\ifx\paragraph\undefined\else
  \let\oldparagraph\paragraph
  \renewcommand{\paragraph}[1]{\oldparagraph{#1}\mbox{}}
\fi
\ifx\subparagraph\undefined\else
  \let\oldsubparagraph\subparagraph
  \renewcommand{\subparagraph}[1]{\oldsubparagraph{#1}\mbox{}}
\fi

\usepackage{color}
\usepackage{fancyvrb}
\newcommand{\VerbBar}{|}
\newcommand{\VERB}{\Verb[commandchars=\\\{\}]}
\DefineVerbatimEnvironment{Highlighting}{Verbatim}{commandchars=\\\{\}}
% Add ',fontsize=\small' for more characters per line
\usepackage{framed}
\definecolor{shadecolor}{RGB}{241,243,245}
\newenvironment{Shaded}{\begin{snugshade}}{\end{snugshade}}
\newcommand{\AlertTok}[1]{\textcolor[rgb]{0.68,0.00,0.00}{#1}}
\newcommand{\AnnotationTok}[1]{\textcolor[rgb]{0.37,0.37,0.37}{#1}}
\newcommand{\AttributeTok}[1]{\textcolor[rgb]{0.40,0.45,0.13}{#1}}
\newcommand{\BaseNTok}[1]{\textcolor[rgb]{0.68,0.00,0.00}{#1}}
\newcommand{\BuiltInTok}[1]{\textcolor[rgb]{0.00,0.23,0.31}{#1}}
\newcommand{\CharTok}[1]{\textcolor[rgb]{0.13,0.47,0.30}{#1}}
\newcommand{\CommentTok}[1]{\textcolor[rgb]{0.37,0.37,0.37}{#1}}
\newcommand{\CommentVarTok}[1]{\textcolor[rgb]{0.37,0.37,0.37}{\textit{#1}}}
\newcommand{\ConstantTok}[1]{\textcolor[rgb]{0.56,0.35,0.01}{#1}}
\newcommand{\ControlFlowTok}[1]{\textcolor[rgb]{0.00,0.23,0.31}{#1}}
\newcommand{\DataTypeTok}[1]{\textcolor[rgb]{0.68,0.00,0.00}{#1}}
\newcommand{\DecValTok}[1]{\textcolor[rgb]{0.68,0.00,0.00}{#1}}
\newcommand{\DocumentationTok}[1]{\textcolor[rgb]{0.37,0.37,0.37}{\textit{#1}}}
\newcommand{\ErrorTok}[1]{\textcolor[rgb]{0.68,0.00,0.00}{#1}}
\newcommand{\ExtensionTok}[1]{\textcolor[rgb]{0.00,0.23,0.31}{#1}}
\newcommand{\FloatTok}[1]{\textcolor[rgb]{0.68,0.00,0.00}{#1}}
\newcommand{\FunctionTok}[1]{\textcolor[rgb]{0.28,0.35,0.67}{#1}}
\newcommand{\ImportTok}[1]{\textcolor[rgb]{0.00,0.46,0.62}{#1}}
\newcommand{\InformationTok}[1]{\textcolor[rgb]{0.37,0.37,0.37}{#1}}
\newcommand{\KeywordTok}[1]{\textcolor[rgb]{0.00,0.23,0.31}{#1}}
\newcommand{\NormalTok}[1]{\textcolor[rgb]{0.00,0.23,0.31}{#1}}
\newcommand{\OperatorTok}[1]{\textcolor[rgb]{0.37,0.37,0.37}{#1}}
\newcommand{\OtherTok}[1]{\textcolor[rgb]{0.00,0.23,0.31}{#1}}
\newcommand{\PreprocessorTok}[1]{\textcolor[rgb]{0.68,0.00,0.00}{#1}}
\newcommand{\RegionMarkerTok}[1]{\textcolor[rgb]{0.00,0.23,0.31}{#1}}
\newcommand{\SpecialCharTok}[1]{\textcolor[rgb]{0.37,0.37,0.37}{#1}}
\newcommand{\SpecialStringTok}[1]{\textcolor[rgb]{0.13,0.47,0.30}{#1}}
\newcommand{\StringTok}[1]{\textcolor[rgb]{0.13,0.47,0.30}{#1}}
\newcommand{\VariableTok}[1]{\textcolor[rgb]{0.07,0.07,0.07}{#1}}
\newcommand{\VerbatimStringTok}[1]{\textcolor[rgb]{0.13,0.47,0.30}{#1}}
\newcommand{\WarningTok}[1]{\textcolor[rgb]{0.37,0.37,0.37}{\textit{#1}}}

\providecommand{\tightlist}{%
  \setlength{\itemsep}{0pt}\setlength{\parskip}{0pt}}\usepackage{longtable,booktabs,array}
\usepackage{calc} % for calculating minipage widths
% Correct order of tables after \paragraph or \subparagraph
\usepackage{etoolbox}
\makeatletter
\patchcmd\longtable{\par}{\if@noskipsec\mbox{}\fi\par}{}{}
\makeatother
% Allow footnotes in longtable head/foot
\IfFileExists{footnotehyper.sty}{\usepackage{footnotehyper}}{\usepackage{footnote}}
\makesavenoteenv{longtable}
\usepackage{graphicx}
\makeatletter
\def\maxwidth{\ifdim\Gin@nat@width>\linewidth\linewidth\else\Gin@nat@width\fi}
\def\maxheight{\ifdim\Gin@nat@height>\textheight\textheight\else\Gin@nat@height\fi}
\makeatother
% Scale images if necessary, so that they will not overflow the page
% margins by default, and it is still possible to overwrite the defaults
% using explicit options in \includegraphics[width, height, ...]{}
\setkeys{Gin}{width=\maxwidth,height=\maxheight,keepaspectratio}
% Set default figure placement to htbp
\makeatletter
\def\fps@figure{htbp}
\makeatother

\KOMAoption{captions}{tableheading}
\makeatletter
\makeatother
\makeatletter
\makeatother
\makeatletter
\@ifpackageloaded{caption}{}{\usepackage{caption}}
\AtBeginDocument{%
\ifdefined\contentsname
  \renewcommand*\contentsname{Table of contents}
\else
  \newcommand\contentsname{Table of contents}
\fi
\ifdefined\listfigurename
  \renewcommand*\listfigurename{List of Figures}
\else
  \newcommand\listfigurename{List of Figures}
\fi
\ifdefined\listtablename
  \renewcommand*\listtablename{List of Tables}
\else
  \newcommand\listtablename{List of Tables}
\fi
\ifdefined\figurename
  \renewcommand*\figurename{Figure}
\else
  \newcommand\figurename{Figure}
\fi
\ifdefined\tablename
  \renewcommand*\tablename{Table}
\else
  \newcommand\tablename{Table}
\fi
}
\@ifpackageloaded{float}{}{\usepackage{float}}
\floatstyle{ruled}
\@ifundefined{c@chapter}{\newfloat{codelisting}{h}{lop}}{\newfloat{codelisting}{h}{lop}[chapter]}
\floatname{codelisting}{Listing}
\newcommand*\listoflistings{\listof{codelisting}{List of Listings}}
\makeatother
\makeatletter
\@ifpackageloaded{caption}{}{\usepackage{caption}}
\@ifpackageloaded{subcaption}{}{\usepackage{subcaption}}
\makeatother
\makeatletter
\@ifpackageloaded{tcolorbox}{}{\usepackage[skins,breakable]{tcolorbox}}
\makeatother
\makeatletter
\@ifundefined{shadecolor}{\definecolor{shadecolor}{rgb}{.97, .97, .97}}
\makeatother
\makeatletter
\makeatother
\makeatletter
\makeatother
\ifLuaTeX
  \usepackage{selnolig}  % disable illegal ligatures
\fi
\IfFileExists{bookmark.sty}{\usepackage{bookmark}}{\usepackage{hyperref}}
\IfFileExists{xurl.sty}{\usepackage{xurl}}{} % add URL line breaks if available
\urlstyle{same} % disable monospaced font for URLs
\hypersetup{
  colorlinks=true,
  linkcolor={blue},
  filecolor={Maroon},
  citecolor={Blue},
  urlcolor={Blue},
  pdfcreator={LaTeX via pandoc}}

\author{}
\date{}

\begin{document}
\ifdefined\Shaded\renewenvironment{Shaded}{\begin{tcolorbox}[borderline west={3pt}{0pt}{shadecolor}, interior hidden, boxrule=0pt, enhanced, sharp corners, breakable, frame hidden]}{\end{tcolorbox}}\fi

\hypertarget{pendahuluan}{%
\section{1 Pendahuluan}\label{pendahuluan}}

\hypertarget{dekompisi-secare-rekursif}{%
\section{2 Dekompisi secare rekursif}\label{dekompisi-secare-rekursif}}

\hypertarget{pendefinisian-operasi-dasar}{%
\subsection{2.1 pendefinisian operasi
dasar}\label{pendefinisian-operasi-dasar}}

\hypertarget{contoh-dnc}{%
\section{3 Contoh DnC}\label{contoh-dnc}}

\hypertarget{binary-search}{%
\subsection{3. Binary Search}\label{binary-search}}

\hypertarget{karatsuba}{%
\subsection{3.1 karatsuba}\label{karatsuba}}

\hypertarget{perkalian-matriks-denga-devide-and-conquer}{%
\section{Perkalian Matriks denga Devide and
Conquer}\label{perkalian-matriks-denga-devide-and-conquer}}

Misalkan kita punya dua matriks A dan B dengan ukuran n x n.~proses
pembagian dilakukan dengan menjadi 4 sub matriks dengan ukuran n/2 x
n/2. Untuk kemudahan kita andaikan \(n = 2^k\) dengan k adalah bilangan
bulat positif. Maka kita dapat menuliskan perkalian matriks A dan B
sebagai berikut:

\[
A=\left(\begin{array}{ll}
A_{11} & A_{12} \\
A_{21} & A_{22}
\end{array}\right), \quad B=\left(\begin{array}{ll}
B_{11} & B_{12} \\
B_{21} & B_{22}
\end{array}\right), \quad C=\left(\begin{array}{ll}
C_{11} & C_{12} \\
C_{21} & C_{22}
\end{array}\right
).
\]

Dengan begitu kita punya matriks \(C\) yang dapat dituliskan sebagai
berikut:

\[
\begin{aligned}
\left(\begin{array}{ll}
C_{11} & C_{12} \\
C_{21} & C_{22}
\end{array}\right) & =\left(\begin{array}{ll}
A_{11} & A_{12} \\
A_{21} & A_{22}
\end{array}\right)\left(\begin{array}{ll}
B_{11} & B_{12} \\
B_{21} & B_{22}
\end{array}\right) \\
& =\left(\begin{array}{ll}
A_{11} \cdot B_{11}+A_{12} \cdot B_{21} & A_{11} \cdot B_{12}+A_{12} \cdot B_{22} \\
A_{21} \cdot B_{11}+A_{22} \cdot B_{21} & A_{21} \cdot B_{12}+A_{22} \cdot B_{22}
\end{array}\right)
\end{aligned}
\] Dengan: \[
\begin{aligned}
& C_{11}=A_{11} \cdot B_{11}+A_{12} \cdot B_{21}, \\
& C_{12}=A_{11} \cdot B_{12}+A_{12} \cdot B_{22}, \\
& C_{21}=A_{21} \cdot B_{11}+A_{22} \cdot B_{21}, \\
& C_{22}=A_{21} \cdot B_{12}+A_{22} \cdot B_{22}
\end{aligned}
\] Perhatikan bahwa kita perlu melakukan 8 kali perkalian matriks
\(n/2 \times n/2\) untuk mendapatkan matriks C. Dengan menggunakan
algoritma DnC. Dengan begitu persamaan rekursif runnig time dari
algoritma perkalian matriks adalah sebagai berikut:

\[T(n)=8 T(n / 2)+\Theta(1))\quad n>1, \quad T(1)=\Theta(1)\] Perhatikan
juga bahwa kompleksitas dari melakukan conquer adalah \(\Theta(1)\)
karena algoritma ini bersifat inplace. Dengan mensubtitusi \(n=2^k\)
maka kita punya persamaan rekurensinya sebagai berikut:
\[ t_k - 8t_{k-1} = 1\] Dan mudah di tunjukkan bahwa perkalian matris
memiliki order of growth \(\Theta(n^3)\)

\begin{Shaded}
\begin{Highlighting}[]
\KeywordTok{def}\NormalTok{ matmul\_dnc(}
\NormalTok{    A: np.ndarray,}
\NormalTok{    B: np.ndarray,}
\NormalTok{    C: np.ndarray,}
\NormalTok{    n: }\BuiltInTok{int} \OperatorTok{=} \VariableTok{None}\NormalTok{,}
\NormalTok{) }\OperatorTok{{-}\textgreater{}}\NormalTok{ np.ndarray:}
    \CommentTok{"""}
\CommentTok{    matmul dnc on nxn matrix it\textquotesingle{}s O(n\^{}3)}
\CommentTok{    """}
    \ControlFlowTok{if}\NormalTok{ n }\OperatorTok{==} \DecValTok{1}\NormalTok{:}
\NormalTok{        C }\OperatorTok{+=}\NormalTok{ A }\OperatorTok{*}\NormalTok{ B}
        \ControlFlowTok{return}
\NormalTok{    n\_init }\OperatorTok{=}\NormalTok{ n}
    \CommentTok{\# if n not power of 2:}
    \ControlFlowTok{if} \KeywordTok{not}\NormalTok{ isPowerOfTwo(n):}
\NormalTok{        n }\OperatorTok{=} \BuiltInTok{max}\NormalTok{(A.shape[}\DecValTok{0}\NormalTok{], A.shape[}\DecValTok{1}\NormalTok{], B.shape[}\DecValTok{0}\NormalTok{], B.shape[}\DecValTok{1}\NormalTok{])}
\NormalTok{        n }\OperatorTok{=} \DecValTok{2} \OperatorTok{**}\NormalTok{ (n }\OperatorTok{{-}} \DecValTok{1}\NormalTok{).bit\_length()}
\NormalTok{        A }\OperatorTok{=}\NormalTok{ np.pad(A, ((}\DecValTok{0}\NormalTok{, n }\OperatorTok{{-}}\NormalTok{ A.shape[}\DecValTok{0}\NormalTok{]), (}\DecValTok{0}\NormalTok{, n }\OperatorTok{{-}}\NormalTok{ A.shape[}\DecValTok{1}\NormalTok{])))}

\NormalTok{        B }\OperatorTok{=}\NormalTok{ np.pad(B, ((}\DecValTok{0}\NormalTok{, n }\OperatorTok{{-}}\NormalTok{ B.shape[}\DecValTok{0}\NormalTok{]), (}\DecValTok{0}\NormalTok{, n }\OperatorTok{{-}}\NormalTok{ B.shape[}\DecValTok{1}\NormalTok{])))}
\NormalTok{        C }\OperatorTok{=}\NormalTok{ np.pad(C, ((}\DecValTok{0}\NormalTok{, n }\OperatorTok{{-}}\NormalTok{ C.shape[}\DecValTok{0}\NormalTok{]), (}\DecValTok{0}\NormalTok{, n }\OperatorTok{{-}}\NormalTok{ C.shape[}\DecValTok{1}\NormalTok{])))}

    \CommentTok{\# conquer}
\NormalTok{    A\_11 }\OperatorTok{=}\NormalTok{ A[: n }\OperatorTok{//} \DecValTok{2}\NormalTok{, : n }\OperatorTok{//} \DecValTok{2}\NormalTok{]}
\NormalTok{    A\_12 }\OperatorTok{=}\NormalTok{ A[: n }\OperatorTok{//} \DecValTok{2}\NormalTok{, n }\OperatorTok{//} \DecValTok{2}\NormalTok{ :]}
\NormalTok{    A\_21 }\OperatorTok{=}\NormalTok{ A[n }\OperatorTok{//} \DecValTok{2}\NormalTok{ :, : n }\OperatorTok{//} \DecValTok{2}\NormalTok{]}
\NormalTok{    A\_22 }\OperatorTok{=}\NormalTok{ A[n }\OperatorTok{//} \DecValTok{2}\NormalTok{ :, n }\OperatorTok{//} \DecValTok{2}\NormalTok{ :]}
\NormalTok{    B\_11 }\OperatorTok{=}\NormalTok{ B[: n }\OperatorTok{//} \DecValTok{2}\NormalTok{, : n }\OperatorTok{//} \DecValTok{2}\NormalTok{]}
\NormalTok{    B\_12 }\OperatorTok{=}\NormalTok{ B[: n }\OperatorTok{//} \DecValTok{2}\NormalTok{, n }\OperatorTok{//} \DecValTok{2}\NormalTok{ :]}
\NormalTok{    B\_21 }\OperatorTok{=}\NormalTok{ B[n }\OperatorTok{//} \DecValTok{2}\NormalTok{ :, : n }\OperatorTok{//} \DecValTok{2}\NormalTok{]}
\NormalTok{    B\_22 }\OperatorTok{=}\NormalTok{ B[n }\OperatorTok{//} \DecValTok{2}\NormalTok{ :, n }\OperatorTok{//} \DecValTok{2}\NormalTok{ :]}
\NormalTok{    C\_11 }\OperatorTok{=}\NormalTok{ C[: n }\OperatorTok{//} \DecValTok{2}\NormalTok{, : n }\OperatorTok{//} \DecValTok{2}\NormalTok{]}
\NormalTok{    C\_12 }\OperatorTok{=}\NormalTok{ C[: n }\OperatorTok{//} \DecValTok{2}\NormalTok{, n }\OperatorTok{//} \DecValTok{2}\NormalTok{ :]}
\NormalTok{    C\_21 }\OperatorTok{=}\NormalTok{ C[n }\OperatorTok{//} \DecValTok{2}\NormalTok{ :, : n }\OperatorTok{//} \DecValTok{2}\NormalTok{]}
\NormalTok{    C\_22 }\OperatorTok{=}\NormalTok{ C[n }\OperatorTok{//} \DecValTok{2}\NormalTok{ :, n }\OperatorTok{//} \DecValTok{2}\NormalTok{ :]}

\NormalTok{    matmul\_dnc(A\_11, B\_11, C\_11, n }\OperatorTok{//} \DecValTok{2}\NormalTok{)}
\NormalTok{    matmul\_dnc(A\_12, B\_21, C\_11, n }\OperatorTok{//} \DecValTok{2}\NormalTok{)}
\NormalTok{    matmul\_dnc(A\_11, B\_12, C\_12, n }\OperatorTok{//} \DecValTok{2}\NormalTok{)}
\NormalTok{    matmul\_dnc(A\_12, B\_22, C\_12, n }\OperatorTok{//} \DecValTok{2}\NormalTok{)}
\NormalTok{    matmul\_dnc(A\_21, B\_11, C\_21, n }\OperatorTok{//} \DecValTok{2}\NormalTok{)}
\NormalTok{    matmul\_dnc(A\_22, B\_21, C\_21, n }\OperatorTok{//} \DecValTok{2}\NormalTok{)}
\NormalTok{    matmul\_dnc(A\_21, B\_12, C\_22, n }\OperatorTok{//} \DecValTok{2}\NormalTok{)}
\NormalTok{    matmul\_dnc(A\_22, B\_22, C\_22, n }\OperatorTok{//} \DecValTok{2}\NormalTok{)}

\NormalTok{    C }\OperatorTok{=}\NormalTok{ np.vstack((np.hstack((C\_11, C\_12)), np.hstack((C\_21, C\_22))))}

    \ControlFlowTok{if} \KeywordTok{not}\NormalTok{ isPowerOfTwo(n\_init):}
\NormalTok{        C }\OperatorTok{=}\NormalTok{ C[:n\_init, :n\_init]}
        \ControlFlowTok{return}\NormalTok{ C}

    \ControlFlowTok{return}\NormalTok{ C}
\end{Highlighting}
\end{Shaded}

\hypertarget{strassen}{%
\subsection{3.2 Strassen}\label{strassen}}

Berdasarkan rangkuman materi brute-force, kita tahu bahwa perkalian
matriks memiliki order of growth \(\Theta(n^3)\). Namun pada tahun 1969,
Volker Strassen menemukan algoritma yang dapat mengurangi order of
growth menjadi \(\Theta({2.807}) = \Theta(n^{\log 7})\). Algoritma ini
menggunakan 7 kali perkalian matriks \(n/2 \times n/2\) untuk
mendapatkan matriks C.

Strassen bisa mengurangi running time dari jumlah perkalian submatriks
yang dibutuhkan menjadi 7, dengan sedikit trade-off pada saat melakukan
conquer, yang menjadi \(\Theta(n^2)\).

Jadi, persamaan rekursif dari algoritma Strassen adalah sebagai berikut:

\[T(n)=7 T(n / 2)+\Theta(n^2))\quad n>1, \quad T(1)=\Theta(1)\]

Dengan master therorem karena \(a=7, b=2, f(n)=\Theta(n^2)\)maka kita
punya kompleksitas waktunya \(\Theta(n^{\log 7})\).

Ide dari penemuan algoritma strassen didasari pada fakta berikut:

\[
x^2-y^2=x^2-x y+x y-y^2 =
x(x-y)+y(x-y)=(x+y)(x-y)
\]

Perhatikan bahwa kita hanya membutuhkan 1 kali perkalian dan 2 kali
operasi penjumlahan atau pengurangan untuk mendapatkan \(x^2-y^2\) pada
form ruas kanan dibandingkan dengan 2 kali perkalian pada form ruas
kiri.

Kita punya algoritma Devide and Conquer strassen sebagai berikut:

\begin{enumerate}
\def\labelenumi{\arabic{enumi}.}
\tightlist
\item
  Jika \(n=1\), matriks masing-masing hanya berisi satu elemen. Lakukan
  perkalian skalar tunggal dan penjumlahan skalar tunggal, yang
  membutuhkan waktu \(\Theta(1)\). Jika tidak, partisi matriks input
  \(A\) dan \(B\) dan matriks output \(C\) menjadi submatriks
  \(n / 2 \times n / 2\), seperti pada persamaan algoritma perkalian
  matriks dengan devide and conquer. Langkah ini membutuhkan waktu
  \(\Theta(1)\).
\item
  Buat matriks \(n / 2 \times n / 2\) \(S_1, S_2, \ldots, S_{10}\),
  masing-masing adalah jumlah atau selisih dari dua submatriks dari
  langkah 1. Buat dan nolkan entri dari tujuh matriks
  \(n / 2 \times n / 2\) \(P_1, P_2, \ldots, P_7\) untuk menampung tujuh
  produk matriks \(n / 2 \times n / 2\). Semua 17 matriks dapat dibuat,
  dan \(P_i\) diinisialisasi, dalam waktu \(\Theta\left(n^2\right)\).
\item
  Dengan menggunakan submatriks dari langkah 1 dan matriks
  \(S_1, S_2, \ldots, S_{10}\) yang dibuat pada langkah 2, secara
  rekursif hitung masing-masing dari tujuh produk matriks
  \(P_1, P_2, \ldots, P_7\), yang membutuhkan waktu \(7 T(n / 2)\).
\item
  Perbarui empat submatriks \(C_{11}, C_{12}, C_{21}, C_{22}\) dari
  matriks hasil \(C\) dengan menambahkan atau mengurangi berbagai
  matriks \(P_i\), yang membutuhkan waktu \(\Theta\left(n^2\right)\).
\end{enumerate}

Berikut adalah matriks antara yang perlu dihitung untuk menghitung
matriks \(C\):

\$\$

\begin{aligned}
& S_1=B_{12}-B_{22}, \\
& S_2=A_{11}+A_{12}, \\
& S_3=A_{21}+A_{22}, \\
& S_4=B_{21}-B_{11}, \\
& S_5=A_{11}+A_{22}, \\
& S_6=B_{11}+B_{22}, \\
& S_7=A_{12}-A_{22}, \\
& S_8=B_{21}+B_{22}, \\
& S_9=A_{11}-A_{21}, \\
& S_{10}=B_{11}+B_{12} .
\end{aligned}

\$\$ perhatikan bahwa Penjumlahan matriks di atas membutuhkan
\(\Theta\left(n^2\right)\) waktu. Dengan menggunakan matriks
\(S_1, S_2, \ldots, S_{10}\), kita dapat menghitung tujuh produk matriks
\(P_1, P_2, \ldots, P_7\) yang dibutuhkan untuk menghitung matriks
\(C\):

\[
\begin{aligned}
& P_1=A_{11} \cdot S_1\left(=A_{11} \cdot B_{12}-A_{11} \cdot B_{22}\right), \\
& P_2=S_2 \cdot B_{22}\left(=A_{11} \cdot B_{22}+A_{12} \cdot B_{22}\right), \\
& P_3=S_3 \cdot B_{11}\left(=A_{21} \cdot B_{11}+A_{22} \cdot B_{11}\right), \\
& P_4=A_{22} \cdot S_4\left(=A_{22} \cdot B_{21}-A_{22} \cdot B_{11}\right), \\
& P_5=S_5 \cdot S_6 \quad\left(=A_{11} \cdot B_{11}+A_{11} \cdot B_{22}+A_{22} \cdot B_{11}+A_{22} \cdot B_{22}\right), \\
& P_6=S_7 \cdot S_8 \quad\left(=A_{12} \cdot B_{21}+A_{12} \cdot B_{22}-A_{22} \cdot B_{21}-A_{22} \cdot B_{22}\right), \\
& P_7=S_9 \cdot S_{10} \quad\left(=A_{11} \cdot B_{11}+A_{11} \cdot B_{12}-A_{21} \cdot B_{11}-A_{21} \cdot B_{12}\right) .
\end{aligned}
\]

Dengan begitu kita punya

\[
C_{11} =C_{11}+P_5+P_4-P_2+P_6
\]

\[
C_{12} = C_{12}+P_1+P_2
\]

\[
C_{21} = C_{21}+P_3+P_4
\]

\[
C_{22} = C_{22}+P_5+P_1-P_3-P_7
\]



\end{document}
